\documentclass[11pt,a4paper]{article}
\usepackage[spanish,es-nodecimaldot]{babel}	% Utilizar español
\usepackage[utf8]{inputenc}					% Caracteres UTF-8
\usepackage{graphicx}						% Imagenes
\usepackage[hidelinks]{hyperref}			% Poner enlaces sin marcarlos en rojo
\usepackage{fancyhdr}						% Modificar encabezados y pies de pagina
\usepackage{float}							% Insertar figuras
\usepackage[textwidth=390pt]{geometry}		% Anchura de la pagina
\usepackage[nottoc]{tocbibind}				% Referencias (no incluir num pagina indice en Indice)
\usepackage{enumitem}						% Permitir enumerate con distintos simbolos
\usepackage[T1]{fontenc}					% Usar textsc en sections
\usepackage{amsmath}						% Símbolos matemáticos

% Comando para poner el nombre de la asignatura
\newcommand{\asignatura}{Programación Técnica y Científica}
\newcommand{\autor}{Vladislav Nikolov Vasilev}
\newcommand{\titulo}{Práctica 1}
\newcommand{\subtitulo}{Práctica sobre poblaciones}

% Configuracion de encabezados y pies de pagina
\pagestyle{fancy}
\lhead{\autor{}}
\rhead{\asignatura{}}
\lfoot{Grado en Ingeniería Informática}
\cfoot{}
\rfoot{\thepage}
\renewcommand{\headrulewidth}{0.4pt}		% Linea cabeza de pagina
\renewcommand{\footrulewidth}{0.4pt}		% Linea pie de pagina

\begin{document}
\pagenumbering{gobble}

% Pagina de titulo
\begin{titlepage}

\begin{minipage}{\textwidth}

\centering

\includegraphics[scale=0.5]{img/ugr.png}\\

\textsc{\Large \asignatura{}\\[0.2cm]}
\textsc{GRADO EN INGENIERÍA INFORMÁTICA}\\[1cm]

\noindent\rule[-1ex]{\textwidth}{1pt}\\[1.5ex]
\textsc{{\Huge \titulo\\[0.5ex]}}
\textsc{{\Large \subtitulo\\}}
\noindent\rule[-1ex]{\textwidth}{2pt}\\[3.5ex]

\end{minipage}

\vspace{0.5cm}

\begin{minipage}{\textwidth}

\centering

\textbf{Autor}\\ {\autor{}}\\[2.5ex]
\textbf{Rama}\\ {Computación y Sistemas Inteligentes}\\[2.5ex]
\vspace{0.3cm}

\includegraphics[scale=0.3]{img/etsiit.jpeg}

\vspace{0.7cm}
\textsc{Escuela Técnica Superior de Ingenierías Informática y de Telecomunicación}\\
\vspace{1cm}
\textsc{Curso 2019-2020}
\end{minipage}
\end{titlepage}

\pagenumbering{arabic}

\setlength{\parskip}{1em}

En esta práctica se han realizado todos los apartados. Existe un \textit{script} principal llamado \textit{main.py},
donde se llaman a todas las funciones necesarias. Estas funciones se encuentran en el directorio \textbf{utils}, formando
un paquete de Python.

A continuación se procede a indicar qué es cada script y en qué apartados se ha utilizado:

\begin{itemize}
	\item \textbf{\textit{dir\_check.py}}: Contiene una función que comprueba si existe el directorio \textbf{resultados},
	y en caso de que no exista, lo crea. Es utilizado por todos los apartados, pero no directamente por \textit{main.py},
	sino de forma auxiliar.
	\item \textbf{\textit{graphics.py}}: Contiene funciones para dibujar gráficos. Utilizado en \textbf{R3}, \textbf{R5}
	y \textbf{R6}.
	\item \textbf{\textit{html\_reader.py}}: Contiene una clase que permite leer los datos de ficheros HTML y obtener información
	de ellos. Se usa en todos los apartados menos \textbf{R1}.
	\item \textbf{\textit{html\_writer.py}}: Contiene una clase que permite escribir los resultados a HTML, incluyendo gráficos
	de ser necesario. Se usa en todos los apartados.
	\item \textbf{\textit{pop\_funcs.py}}: Contiene funciones para realizar operaciones sobre la población, como obtener
	las variaciones, calcular la población por comunidad, determianr las comunidades con mayor población o ver si los
	valores de variación de dos tablas son iguales. Se usa en todos los apartados.
	\item \textbf{\textit{read\_csv.py}}: Contiene una función para leer archivos CSV y convertirlos a diccionarios
	de diccionarios, simulando de esta forma el funcionamiento de \textit{pandas}. Utilizado en \textbf{R1}.
\end{itemize}





\end{document}

